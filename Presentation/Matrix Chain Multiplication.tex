\documentclass[algorithm,pgfplots]{cuzbeamer}
\usepackage[
backend=biber,
style=alphabetic,
sorting=ynt
]{biblatex}
\usepackage{hyperref}
\addbibresource{mybibliography.bib} %Imports bibliography file
\usepackage[scale=2]{ccicons}
%\usepackage{graphicx,xcolor}
\usepackage{tikz, stackengine}
%\usetikzlibrary{decorations.text}
%\usetikzlibrary{shapes}
%\usetikzlibrary{overlay-beamer-styles}
%\usetikzlibrary{matrix}
\usepackage{amsmath}
%\usepackage[table,xcdraw]{xcolor}


\def\tikzoverlay{%
   \tikz[baseline,overlay]
}%

\newcommand{\tikzmark}[1]{\tikz[overlay,remember picture] \node (#1) {};}

\newcommand*{\LabelText}[3]{%
\tikzmark{a}#1\tikzmark{b}%
\begin{tikzpicture}[overlay,remember picture]
    \path (a.north) -- (b.north) node [yshift=3.0ex, midway,rectangle,draw=#3,rounded corners=2pt,inner sep=1pt,fill=-#3]  {#2\strut};
\end{tikzpicture}%
}


\begin{document}
    \title{Matrix Chain Multiplication}
\subtitle{CSE 300: Technical Writing and Presentation}
\date{August 30, 2022}
\author{1805002 - A. H. M. Osama Haque\\ 1805009 -  Md. Zarzees Uddin}

    \maketitle

    \begin{frame}{Contents}
        \setbeamertemplate{section in toc}[sections numbered]
        \tableofcontents
    \end{frame}

    \section{Description}

    \begin{frame}
    \printbibliography
        \onslide<1->\begin{block}{\large{\textbf{Aim}}}
            \begin{leftbar}
           \textbf{ To find the most efficient way to multiply a sequence of matrices}
            \end{leftbar}
        \end{block}
        \onslide<2->
        A : 5x20 \hspace{1cm}  B: 20x10 \hspace{1cm} C: 10x50
    
    \onslide<3->
    
    \begin{block}{\large{\textbf{Compute ABC}}}
    \begin{itemize}[(1)]
    
    \item<1-> \onslide<4-> (AB)C   \onslide<6->{\quad 5*20*10 + 5*10*50} \onslide<7->{ = \textcolor<8>{green}{3500} operations}
    \item<1->\onslide<5->  A(BC) \onslide<6-> {\quad 20*10*50 + 5*20*50} \onslide<7->{
    = \textcolor<8>{red}{15000} operations}
    \end{itemize}
    \end{block}   
    
    \end{frame}

    \section{Brute Force Algorithm}

    \begin{frame}
    \begin{block}{\large{\textcolor{cyan}{\textbf{Brute Force Approach :}}}}
    \\
      \vspace{0.75cm}
        
        \begin{itemize}
            \item \onslide<2->
        Consider all possible cases. \\
        \item \onslide<3->
        Not a good idea since the number of operations increases exponentially with the number of matrices to multiply. \\
        \item \onslide<4-> \textcolor{red}{EXPONENTIAL-TIME Algorithm} \\
        \textcolor{green}{Very slow and not practical} \\
        \end{itemize}  
    \end{block}
    \end{frame}
    
    \begin{frame}
    \centering {\includegraphics[scale=0.05]{think.png}}\\
    \vspace{1cm}
    \Large \textcolor{cyan}{\textbf{Effective Solution?}}
    
        
    \end{frame}

    \section{Dynamic Programming Algorithm}
    
   \begin{frame}
   \begin{block}{\large{\textcolor{cyan}{\textbf{Dynamic Programming Approach}}}}\vspace{1cm}
   
       To compute A$_{1}$A$_{2}$...A$_{n}$ \\
       n : \# matrices to multiply \\ 
       \\
       A$_{i}$ $\rightarrow$ (d$_{i-1}$ x d$_{i}$)  \text{ matrix} \\
       A$_{i}$A$_{i+1}$...A$_{j}$ $\rightarrow$ (d$_{i-1}$ x d$_{j}$)  \text{ matrix}\\
              m(i,j) : Min number of operations required to compute product of A$_i$ to A$_j$ 
   \end{block}
   \end{frame}

    \begin{frame}
        \begin{block}{\textcolor{cyan}{\large{\textbf{Algorithm}}}}
        \vspace{0.75cm}
        \begin{itemize}
            \onslide<1->\item \textcolor{orange}{\textbf{Step 1:}} $m(i,j) = \times $ , $\forall_{i>j}$\\
            \onslide<2->\item \textcolor{orange}{\textbf{Step 2:}} 
          \forall$_{1 \leq i \leq j \leq n}$ \\
            % \quad if i = j,\\
            % \hspace{1cm}  m(i,j) = 0\\
            % \quad else \\
            % \hspace{1cm}
            % m(i,j) = min \{ m(i,k) + m(k+1, j) + d$_{i-1}$ d$_{k}$ d$_{j}$ \} \\
            m(i,j) =
            \begin{cases}
                & 0 $\quad\quad\quad\quad\quad$ \text{if }  i=j\\
                &  \text{min}$_{i\le k < j}$ \{ m(i,k) + m(k+1, j) + d$_{i-1}$d$_k$d$_j$ \} \ \text{if $i \ne j$} \\
                 
            \end{cases}
            
        \quad \onslide<3->\item \textcolor{orange}{\textbf{Step 3:}} Return $m(1,n)$
        \end{itemize}
        \end{block}
    \end{frame}
    \begin{frame}
    \begin{block}{\large{\textcolor{cyan}{\textbf{Example}}}}
    \vspace{0.7cm}
    Let us consider an example, \\
    A $\rightarrow$ 40\times20 \\
    B $\rightarrow$ 20\times30 \\
    C $\rightarrow$ 30\times10 \\
    D $\rightarrow$ 10\times30 \\
    
    We create a $4\times4$ table for implementing the algorithm
    
    
    \end{block}
        
    \end{frame}
    
    \begin{frame}
        \begin{block}{\large{\textcolor{cyan}{\textbf{Example}}}}
   \vspace{0.25 cm}
   \begin{tabular}{ccccccc}
        A & \times & B & \times & C & \times & D  \\
        40\times20 && 20\times30 && 30\times10 && 10\times30 \\
   \end{tabular}
        
\begin{columns}
\column{0.5\textwidth}
\begin{table}[]
\begin{tabular}{c|cccc}
{\color[HTML]{FFFFFF} i/j} & {\color[HTML]{FFFFFF} 0} & {\color[HTML]{FFFFFF} 1}     & {\color[HTML]{FFFFFF} 2}     & {\color[HTML]{FFFFFF} 3}     \\ \hline
0   &
\onslide<2->{\color[HTML]{FFFFFF}
\textcolor<4,10>{green}{0}}&  \onslide<3->{\color[HTML]{FFFFFF} \textcolor<5,11>{green}{24000}} & \onslide<6->{\color[HTML]{FFFFFF} \textcolor<6,12>{green}{14000}} & {\onslide<13->{\color[HTML]{FFFFFF} \textcolor{green}{26000}}} \\
{1}   & {x} &  \onslide<2->{\color[HTML]{FFFFFF}
\textcolor<7>{green}{0}}     & 
\onslide<3->{\color[HTML]{FFFFFF}\textcolor<4,8>{green}{6000}}  & \onslide<9->{\color[HTML]{FFFFFF}  \textcolor<9,10>{green}{12000}} \\
{2}   & {\color[HTML]{FFFFFF} x} & {\color[HTML]{FFFFFF} x}     & \onslide<2->{\color[HTML]{FFFFFF}
\textcolor<5>{green}{0}}     & \onslide<3->{\color[HTML]{FFFFFF}  \textcolor<7,11>{green}{9000}}  \\
{3}   & {\color[HTML]{FFFFFF} x} & {\color[HTML]{FFFFFF} x}     & {\color[HTML]{FFFFFF} x}     & \onslide<2->{\color[HTML]{FFFFFF}  \textcolor<8,12>{green}{0}}  
\end{tabular}
\end{table}
\column{0.5\textwidth}
\onslide<2-> \textcolor{orange}{\textbf{Step 1 :}}
Fill the table for i = j \textcolor<2>{green}{(All zero values)}\\ 

\onslide<3->
\textcolor{orange}{\textbf{Step 2 :}} \\ \textcolor<3>{green}{m(0,1) = 40\times20\times30=24000}\\
\textcolor<3>{green}{m(1,2) =20\times30\times10=6000}\\
\textcolor<3>{green}{m(2,3) =30\times10\times30=9000}\\
\onslide<6->
\textcolor<6>{green}{m(0,2) = min(14000, 36000) = 14000}\\
\onslide<9->
\textcolor<9>{green}{m(1,3) = 12000}\\
\onslide<13->
\textcolor<13>{green}{m(0,3) = min(26000, 38000) = 26000} \\

\onslide<14->\textcolor{orange}{\textbf{Step 3 :}} return 26000\\
            
\end{columns}
        \end{block}
    \end{frame}
\begin{frame}
    \begin{block}{\large{\textcolor{cyan}{\textbf{Example}}}}
    \vspace{1cm}
    So, the resulting sequence : \\
    \onslide<1->(A B C) D \\
    \onslide<2-> $\rightarrow$ \textcolor{yellow}{(A (B C)) D} \quad [keeping track of the variable "k"] \\
    \onslide<3->And the minimum number of multiplication operation = \textcolor{yellow}{26000}
    
    
    \end{block}
        
\end{frame}    
    
    
\begin{frame}
\begin{block}{\large{\textcolor{cyan}{\textbf{Matrix Chain Multiplication}}}}
\vspace{0.5cm}
\centering
\onslide<1->{Time Complexity : \textbf{$O(n^3)$}}\\
\onslide<2->{Seems most optimal?}
\onslide<3->\textbf{\alert {  No!!!}} \\
\onslide<4->
{\includegraphics[scale=0.25]{Paper.png}} \\
\onslide<5>{An algorithm published by \textbf{Hu} and \textbf{Shing} (1982) achieves $O(n log n)$ computational complexity \\
They reduced the problem into \alert{Triangulation of a Regular Polygon.}}
\end{block}

\end{frame} 

\begin{frame}
\begin{block}{\large{\textcolor{cyan}{\textbf{References}}}}
\vspace{3cm}
\begin{itemize}
    \item \href{https://en.wikipedia.org/wiki/Matrix_chain_multiplication}{Wikipedia/Matrix Chain Multiplication}
    \item Introduction to Algorithms by Cormen
    \item stackoverflow.com
\end{itemize}
\end{block}

    
\end{frame}

\end{document}
