\documentclass{article}
\usepackage[utf8]{inputenc}

\usepackage{amsmath}
\usepackage{multicol}
\usepackage{multirow}
\usepackage{graphicx}
\usepackage{array}

\title{practice}
\author{Osama Haque}
\date{May 2022}

\begin{document}


\maketitle

\tableofcontents %this command creates the table of contents with all numbered sections, subsections, etc.
\pagebreak %This will force the rest of the document to start in another page.

\section{Introduction}
\LaTeX

\section{Mathematical Equations}
$
{y} = 1 - \frac{f^n[\frac{s.1}{f} + {(\frac{20}{f})}^w ]}{20^n}
$
%Equation 1
\begin{equation}
    {y} = 1 - \frac{f^n[\frac{s.1}{f} + {(\frac{20}{f})}^w ]}{20^n}
\end{equation}

$
\frac{df}{dt} = \lim_{h\to 0} \frac{f(t+h)-f(t) }{h}
$

%Equation 2
\begin{equation}
    \frac{df}{dt} = \lim_{h\to 0} \frac{f(t+h)-f(t) }{h}
\end{equation}

%Equation 3
\begin{equation}
    a = \max_{x \in \mathbf{R}}f(x) 
\end{equation}

%Equation 4
\begin{equation}
   \int_{-\infty}^{\infty} e^{-x^2} dx = \sqrt{\pi}
\end{equation}

%Equation 5
\begin{equation}
    f(x) = a_0 + \sum_{n=1}^\infty(a_n\cos{\frac{n \pi x}{L}} + b_n\sin{\frac{n \pi x}{L}})
\end{equation}

%Equation 6
\begin{equation}
    x = \frac{-b \pm \sqrt{b^2 - 4ac}}{2a}   %pm = plus-minus sign
\end{equation}

\subsection{Align}
\begin{align}
    f(x) &= x^2-1 \nonumber \\
     &= (x-1)(x+1)
\end{align}




\begin{tabular}{|l|l|}
\hline
b &   \\ \cline{1-1}
a & c \\
a &   \\ \cline{2-2} 
a & e \\ \hline
\end{tabular}

\begin{tabular}{|r|r|}
    \hline
    b &  \multirow{3}{*}{c} \\
    \cline{1-1}
    a &    \\
    a &    \\
    \cline{2-2}
    a & e  \\
    \hline
    
\end{tabular}



\begin{tabular}{ | p{15em} | p{1cm}| p{5cm} | } 
  \hline
  cell1 dummy text dummy text dummy text& cell2 & cell3 \\ 
  \hline
  cell1 dummy text dummy text dummy text & cell5 & cell6 \\ 
  \hline
  cell7 & cell8 & cell9 \\ 
  \hline
\end{tabular}



\section{emphasis}
\emph{This sentence is an example of emphasis. Even though it is italic in this example, using another package might change it to something else, like underlined text.  \emph{Also, emphasis can be nested.} The previous sentence was an example of nested emphasis.} 

\section*{Unnumbered Section Example}
This section should not be numbered. Using * after the section specifier prevents this numbering. This will NOT show up in the Table of Contents.
\LARGE{This sentence is LARGE.}
\Huge{This sentence is HUGE.}

\section{Class tables}

\subsection{basic Tables}

\begin{tabular}{|m{3cm}| m{2cm}|m{2cm}|}
    \hline
    \textbf{x} & \textbf{y} \textbf{(in centimeter)} & \textbf{h} \\
    \hline
    {1} & 2 & h\\
    \hline
    200 & 400 & 87 \\
    \hline
    
\end{tabular}


\begin{tabular}{|c|c|c|c|}
    \hline
	1 & 2 & 3 & 4 \\ 
	\cline{1-1}\cline{3-4}
	1 & 2 & 3 & 4  \\
	\hline
	\multicolumn{2}{|c|}{text} & 4 & 5 \\
	\hline
	\multirow{3}{*}{text} & 2 & 3 & 4 \\
	\cline{2-4}
	& 2 & 3 & 4 \\
	& d & s & r \\
	\hline
	\multicolumn{3}{|c|}{\multirow{3}{*}{text afdsf}} & 5 \\
	\multicolumn{3}{|c|}{} & 7 \\
	\multicolumn{3}{|c|}{} & 7 \\
	\hline

\end{tabular}

\subsection{Multi-Column}
\begin{tabular}{|c|ccc|}
    \hline
    1 & \multicolumn{3}{c|}{text} \\
    \cline{2-4}
    1 & 2 & 3 & 4 \\

    \hline
\end{tabular}

\subsection{Cell-wise align}
\begin{tabular}{|c | c|}
    \hline
    x & y \\
    \hline
    1 & \multicolumn{1}{l|}{2} \\
    200 & 400 \\
    \hline
\end{tabular}

\begin{tabular}{|c c | c |}
    \hline
     1 & 2 & 3 \\
     \hline
     \multicolumn{2}{|c|}{\multirow{2}{*}{text}} & 5  \\
     \multicolumn{2}{|c|}{} & 7 \\
     \hline
\end{tabular}

\begin{tabular}{|c|c|c|c|}
\hline
\multicolumn{1}{|c|}{\multirow{2}{*}{Si. No.}} & \multicolumn{2}{c}{Reading} & \multicolumn{1}{|c|}{\multirow{2}{*}{Diff}} \\
\cline{2-3}
 & initial & final & \\
 \hline
\end{tabular}


\section{Inside Table scope}

\begin{table}[tbp]  %t = top of the page
                    %b = bottom of the page
                    %p = center of a new page
    \centering
    \begin{tabular}{|c c|}
        \hline
        x & y \\
        \hline
        1 & 2 \\
        200 & 400 \\
        \hline
    \end{tabular}
    \caption{New Table}
    \label{tab:my_label}
\end{table}


\section{from samee}

\subsection{}
\begin{tabular}{| c|c | c | c|}
\hline
     \multirow{2}{*}{Si. No.} & \multicolumn{2}{c|}{Reading} & \multirow{2}{*}{Difference}   \\
     \cline{2-3}
   & initial & final &  \\
\hline
\end{tabular}

\subsection{}
\begin{tabular}{| c|c | c | c|}
\hline
    a & b & c & \multirow{2}{*}{d} \\
    \cline{1-3}
    \multicolumn{3}{|c|}{e} & \\
    \hline
    f & g & \multicolumn{2}{c|}{h} \\
    \hline
\end{tabular}

\subsection{}
\begin{tabular}{|c|c|c|}
\hline
    \multicolumn{2}{|c|}{\multirow{2}{*}{a}}     & \multirow{6}{*}{b}   \\
    \multicolumn{2}{|c|}{}                       &                      \\ \cline{1-2}
    {c} & \multirow{2}{*}{e}                    &                       \\ \cline{1-1}
    {d}     &           &                           \\ \cline{1-2}
\multicolumn{2}{|c|}{\multirow{4}{*}{f}}     &                                         \\
\multicolumn{2}{|c|}{}                       &                                         \\ \cline{3-3} 
\multicolumn{2}{|c|}{}                       & {\multirow{2}{*}{g}} \\
\multicolumn{2}{|c|}{}                       &                  \\ \hline
\end{tabular}

\end{document}




